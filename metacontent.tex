
\maketitle

\begin{abstract}
We present a systematic study of algebraic operations on the set of all complex-valued multiplicative functions, unifying and generalizing many results in the literature. The principal binary operations we use in our analysis are (1) Dirichlet convolution, (2) unitary convolution, (3) natural product, and (4) a new operation which we call the tensor product, which is closely related to the Rankin-Selberg convolution of L-functions. The principal unary operations are (1) the higher norm operators of Redmond and Sivaramakrishnan, which are also related to Weil restriction on varieties over finite fields, (2) precomposition with a power function, (3) the $k$'th convolute, and (4) a new operator related to base change of varieties over finite fields. We also introduce two auxiliary unary operations called the Bell derivative and the Bell antiderivative (they are inverse to each other). The Bell derivative generalizes the notion of a \emph{unitary analogue} of a multiplicative function. Based on many examples, it also seems like the Bell derivative connects point counting over finite fields to point counting using modular arithmetic, in cases where the motives of the equations we consider are mixed Tate motives.

Essentially all operations on multiplicative functions appearing in the number theory literature can be expressed in terms of our eight fundamental operations. We illustrate this with many examples.

In order to express our structural results on the eight operations, we introduce axioms for a new algebraic structure called a double Adams algebra, which is essentially the same thing as a lambda-ring (or special lambda-ring in the sense of Grothendieck) together with a right inverse to each Adams operation, where these right inverses are required to be (not necessarily unital) ring homomorphisms.

Our first main result say that the four principal binary operations and the four principal unary operation can be organized into two separate double Adams algebra structures on the set of all multiplicative functions. These two structures are distinct but isomorphic, and the Bell derivative provides an explicit isomorphism between them.

We define an important class of multiplicative functions called the \emph{rational} multiplicative functions. All the multiplicative functions of elementary number theory fall into this class, and the same is true of every multiplicative function given by the Fourier coefficients of a motivic or an automorphic L-function. We also introduce a topology (the Bell topology) on the set of all multiplicative functions for which the set of rational functions is a dense subset.

In order to do efficient and explicit computations with the eight operations, we develop a correspondence between rational multiplicative functions and a new construct called Tannakian symbols. The power of this new symbolic language is illustrated by numerous applications, including a Sage program for automated discovery and automated proof of many new identities between multiplicative functions. 

The motivation behind Tannakian symbols is a long-term program aiming to develop methods for explicit computations in Grothendieck rings of Tannakian categories. These applications are not fully developed in the current paper, but we give some hints and some conjectures. The computational methods may also be useful in $K$-theory and other settings where lambda-rings appear naturally.


\end{abstract}

\newpage
\setcounter{tocdepth}{2}
\tableofcontents

\newpage
