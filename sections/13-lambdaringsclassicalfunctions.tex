
\section{Some small lambda-rings generated by classical functions}

The goal of this section is apply the functorial language of the previous section to describe the structure of lambda-rings generated by some of the most well-known classical multiplicative functions. We do this only for the standard lambda-ring structure, since this is the most interesting one (explain link to Grothendieck rings, and that this is a practice problem). \emph{What about the other three? Would it be possible to the same for them in the future?}


\subsection{Examples of lambda-rings}

Introduce notation for lambda-rings generated by a set of multiplicative functions? Or simply discuss the need for such notation. Mention perhaps the lambda-rings generated by a function from $\mathbb{P}$ to $\mathbb{C}^{\times}$. Suggestion: $\Lambda[\mathbf{1}]$, $\Lambda[id]$, $\Lambda[?]$, $\Lambda[p, -1]$ etc. 

Give an overview of these small lambda-rings and their inclusion relations. State in particular all possible choices of generators of for example $\Lambda[p]$, among the classical multiplicative functions. Explain also the relation to Dirichlet series.

\begin{proposition}
The following lambda-rings are equal:
\begin{itemize}
\item The image of $\iota_{\mathcal{F}}$, for $\mathcal{F} = \{ p \} \cup W_k$. 
\item ?
\end{itemize}

\end{proposition}



\subsection{Sort out}

Example: Functions from primes to complex numbers depending only on the congruence class of the prime.

Add comment saying that we are working on Dirichlet characters. Add perhaps Langlands example, or the example from Sutherland's Arizona notes.


Plan: Define classical multiplicative functions as everything coming from the monoid generated by prime powers of non-negative exponent and roots of of unity. We get $TS(\mathbb{N}_0^{+} \times W_{\infty})$. Discuss then in minor remarks some sub-lambda-rings of this, with a hint that a complete classification might be possible, and discuss also some extensions, such as TS of the multiplicative monoid $\mathbb{Z}[p] \ \{0\}$, which would give a much larger class of functions, but only add few extra elementary examples, such as the core function and the function with a 2 in the symbol. Use notation $Mult_{elem}$ but describe only with words the classes and generators discussion (but compare with Dynkin diagrams). 

Note also the connection with the Riemann zeta function, which should make the elementary functions a very natural class of things to study. Work out which shifts and scalings we get before writing the rest of the section.




Proposition on correspondence with Dirichlet series expressed via zeta.

We could say something about other classes of multiplicative functions that admit lambda-ring structures.  
\begin{itemize}
\item All Dirichlet characters?
\item Dirichlet characters for a fixed modulus/conductor? Other classes dependent only on congruence class of p?
\end{itemize}


