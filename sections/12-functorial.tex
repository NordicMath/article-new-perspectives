
\section{The functorial theory of Tannakian symbols}


\subsection{\textbf{TS} as a functor of commutative monoids}

\subsection{Uniformity conditions}
\todo{what in the world is this???}
Recall examples of uniform and almost uniform functions. Want to formalize this.

\begin{definition}
The \defhl{prime support} of a multiplicative function is the set of primes for which the Bell series $f_p(t)$ is not equal to 1. A function has \defhl{finite prime support} if its prime support is a finite set. 
\end{definition}

\begin{example}
gcd functions, stuff coming from varieties over a finite field.
\end{example}

Consider the set $\C \backslash \{ 0\}$ of nonzero complex numbers. It is a monoid under multiplication, and hence the set of functions $(\C \backslash \{ 0 \})^{\mathbb{P}}$ is also a monoid under (pointwise) multiplication.

Let $\mathcal{F}$ be a set of functions from the prime numbers to the nonzero complex numbers, i.e. any subset of $(\C \backslash \{ 0 \})^{\mathbb{P}}$. Let $\langle \mathcal{F} \rangle$ be the multiplicative monoid generated by $\mathcal{F}$. Then $\TS(\langle \mathcal{F} \rangle)$ is a lambda-ring. We define a function
$$ \iota_{\mathcal{F}} : \TS(\langle \mathcal{F} \rangle) \to Mult_{rat}(\C)  $$
in the obvious way. It is clear that $\iota_{\mathcal{F}}$ is injective.



\begin{definition}
Let $\mathcal{F}$ be as above. We say that a multiplicative function $f$ is \defhl{uniform with respect to $\mathcal{F}$} if it is in the image of $\iota_{\mathcal{F}}$. We say that $f$ is \defhl{almost uniform with respect to $\mathcal{F}$} if there exists a multiplicative function $g$ which is uniform with respect to $\mathcal{F}$ and is such that $f \ominus g$ has finite prime support. 
\end{definition}

\begin{example}
Add here many examples of uniform and almost uniform functions with respect to various $\mathcal{F}$ \emph{after all examples have been completed}.
\end{example}

Better language! Let $F$ be a class of functions from the primes to the complex numbers, which form a monoid under multiplicatation (Def: Monoidal class of functions? Say that a multiplicative function is uniform with respect to $F$ if bla bla, and say that it is almost uniform with respect to $F$ if bla bla. In both cases we get lambda-rings with respect to the standard structure. Add also the idea of a group action at some level, and the idea of taking invariant symbols to get to some of the smaller subclasses.


\subsection{\textbf{TS} as a trifunctor}

Notation: $\TS_K^U(M)$???
\todo{what was the third parameter?}
\subsection{A third meta-mathematical claim}

Our third and final meta-mathematical claim comes with less evidence than the previous two. We \todo{well...} believe that

\begin{quote}
\textbf{Any lambda-ring appearing in nature is isomorphic to a sub-lambda-rings of $\TS^U_K(M)$ for some choice of $U$, $K$ and $M$.}
\end{quote}

\todo{How about C? Q? Don't those appear in nature?}

Let us consider a few examples.

\begin{example}
The representation ring of a finite group.  
\end{example}

\begin{example}
We are not able to give a precise comparison theorem, but it seems like the above claim, applied to the Grothendieck ring of motives over a finite field, is essentially equivalent to the Tate conjecture.
\end{example}

\begin{example}
Could add motivation from Dynkin diagrams.
\end{example}

\begin{example}
Rational Witt vectors.
\end{example}

S and K says that over a field, rational Witt vectors have an elementary description as the Galois-invariant part of a group algebra. Can we say something meaningful about what happens for other rings, and relate this to Tannakian symbols? For a ring of integers, we can use Galois invariants of the units of its integral closure, I believe!

References for Witt vectors:
The paper of Hesselholt, the note of Lenstra (\url{https://math.berkeley.edu/~hwl/papers/witt.pdf}), the book of Yau. Probably others as well.

\subsection{Complexity measures for lambda-rings}

Introduce $G$-complexity, $U$-complexity and $P$-complexity.



\subsection{$\TS$ as a monad}

Explain here TS as a monad.
