
\section{Lambda-rings}

We review the notion of a lambda-ring (originally called \emph{special lambda-ring} by Grothendieck. We also define what it means for a lambda-ring to \emph{admit $\Phi$-operations}, a notion not previously considered in the literature. Finally, we give several key examples, most of them needed in later sections.


\subsection{Review of definitions}

Let $R$ be a unital commutative ring. We want to explain what is meant by a \emph{lambda-ring structure} on $R$. There are two ways of doing this. The first definition is valid for any $R$, but is rather complicated. The second definition, which is much simpler, is valid only when $R$ is torsion-free.

Since all rings in this paper will be torsion-free, we give only the second definition here, and refer the reader to the book of Yau \cite{Yau} or the Encyclopedia of Mathematics for the more general one.

\begin{definition}
A ring is called \defhl{torsion-free} if for every positive integer $n$ and every element $x \in R$, the relation $nx=0$ implies $x=0$.
\end{definition}

\begin{definition}
Let $R$ be a unital commutative ring. A \defhl{lambda-ring structure} on $R$ is a sequence $\Psi^1$, $\Psi^2$, $\Psi^3$, ... of ring homomorphisms from $R$ to $R$, satisfying the axioms:
\begin{enumerate}
\item
\item
\end{enumerate}

\end{definition}

\begin{definition}
A \defhl{lambda-ring} is a commutative ring $R$ together with a lambda-ring structure on $R$.
\end{definition}

The operations $\Psi^k$ (for $k \in \mathbb{N}$) are called Adams operations. From the Adams operations, one may define other sequences of operations, namely:
\begin{itemize}
\item Lambda-operations: $\Lambda^1$, $\Lambda^2$, $\Lambda^3$, \ldots
\item Symmetric power operations: $Sym^1$, $Sym^2$, $Sym^3$, \ldots
\item Gamma operations: $\Gamma^1$, $\Gamma^2$, $\Gamma^3$, \ldots
\end{itemize}
Each of these is a function from $R$ to $R$ \emph{but} unlike the Adams operations, they are not ring homomorphisms except in degenerate cases. The definitions are omitted for now, but can be found in the references given above.

\begin{definition}
Let $R$ be a lambda-ring with Adams operations $\Psi^k$. A \defhl{family of $\Phi$-operations} on $R$ is a sequence $\Phi^k$ ($k = 1, 2, 3, \ldots$) of functions from $R$ to $R$, satisfying these axioms:
\begin{enumerate}
\item $\Phi^k$ is a section of $\Psi^k$, i.e. $\Psi^k(\Phi^k(x)) = x$ for all $x \in R$.
\item $\Phi^k$ is a not-necessarily-unital ring homomorphism, i.e. for all $x, y \in R$, we have
$$ \Phi^k(x+y) = \Phi^k(x) + \Phi^k(y) \quad \textrm{and} \quad \Phi^k(x\cdot y) = \Phi^k(x)\cdot \Phi^k(y) $$
\item For all $k$ and $m$, we have $\Phi^k \circ \Phi^m = \Phi^{km}$
\end{enumerate}
If such a family exists, we say that the lambda-ring $R$ \defhl{admits $\Phi$-operations}.
\end{definition}


\subsection{Examples of lambda-rings}




\begin{example}
Let $G$ be a finite group, and let $R(G)$ be the complex representation ring of $G$.

Take a specific example of a group and explain character table and power maps. Relate this to Bell tables.
\end{example}

\begin{example}
Topological K-theory.
\end{example}

\begin{example}
Algebraic K-theory and motivic cohomology.
\end{example}

\begin{example}
Grothendieck ring of a symmetric monoidal abelian category (in particular a Tannakian category).
\end{example}

\begin{example}
I think there are more examples in Yau and/or elsewhere. Witt vectors? Polynomial rings? Symmetric operations?
\end{example}
