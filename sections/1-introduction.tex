
\section{Introduction}

%Multiplicative functions are ubiquitous in modern number theory. Well-known examples include the Euler $\varphi$ function, the sum-of-divisors function, and the M\"obius function. In this paper, we introduce new algebraic structures on various classes of multiplicative functions, going beyond the standard abelian group structure provided by Dirichlet convolution and the standard commutative monoid structure given by pointwise multiplication of functions. In particular, we prove that the class of multiplicative functions with rational Bell series (see Definition ...) is a lambda-ring (in the sense of Grothendieck), in which addition is given by Dirichlet convolution, multiplication is given by a certain ``deformation" of the pointwise multiplication, and the Adams operations generalize the higher norm operators of Redmond and Sivaramakrishnan (give ref).

%The key tool in this analysis is a correspondence between multiplicative functions satisfying certain conditions, and a device called Tannakian symbols.

%In addition to their intrinsic interest, these new algebraic structures yield very quick proofs of many classical identities between multiplicative functions.


%\subsection{Main results}

%Can we define Tannakian symbols here and summarize the main results?

%\todo[inline]{I think we can put some examples here, rather than trying to formulate main results in a precise way? Or could we give an diagrammatic overview over the various small lambda-rings generated by different functions? Maybe a table of functions with associated symbols?}
%In order to define and prove results about our new operations, we begin by setting up a correspondence between multiplicative functions and something we call Tannakian symbols.


\subsection{Summary}



\newpage
\thispagestyle{empty}
\adjustbox{center}{
\begin{tikzpicture}[block/.style={draw=black,rounded corners=2ex,minimum width=7cm,minimum height=7cm,outer sep=5pt,inner sep=10pt},content/.style={inner sep=10pt,align=left,below right,text width=7cm},title/.style={align=center,below,font=\large,text height=0.5cm}]

  \node(atcenter) at (0,0) {}; 
  \node(multratbox)   [block,right=1cm of atcenter]{};
  \node(multratcircle)[block,left=1cm  of atcenter]{};
  \node(varieties)    [block,above=2cm of multratbox,minimum height=5cm]{};
  \node(lfunc)        [block,above=2cm of multratcircle,minimum height=5cm]{};
  \node(multcbox)     [block,below=2cm of multratbox,minimum height=6cm]{};
  \node(multccircle)  [block,below=2cm of multratcircle, minimum height=6cm]{};
  
  \node[title]   at (varieties.north){\uline{Varieties}};
  \node[content] at (varieties.north west){\ \\ \ \\ \ \\
  Varieties \\
  I'm right here!
  };
  
  \node[title]   at (lfunc.north){\uline{L-functions}};
  \node[content] at (lfunc.north west){\ \\ \ \\ \ \\
  L-functions \\
  I'm over here!
  };
  
  \node[title]   at (multratcircle.north){\uline{$Mult_{rat}(\C)$}};
  \node[content] at (multratcircle.north west){\ \\ \ \\ \ \\
  MultRat \\
  I'm just here!
  \begin{itemize}
  \item test1 \\
  \item test2
  \end{itemize}
  };
  
  \node[title]   at (multratbox.north){\uline{$Mult_{rat}(\C)$}};
  \node[content] at (multratbox.north west){\ \\ \ \\ \ \\
  MultRat \\
  I'm here! \\
  I'm here! \\
  I'm here! \\
  I'm here! \\
  I'm here! \\
  I'm here!
  };
  
  \node[title]   at (multccircle.north){\uline{$Mult(\C)$}};
  \node[content] at (multccircle.north west){\ \\ \ \\ \ \\
  MultC \\
  I love you all!
  };
  
  \node[title]   at (multcbox.north){\uline{$Mult(\C)$}};
  \node[content] at (multcbox.north west){\ \\ \ \\ \ \\ 
  MultC \\
  I am over here!
  };
  
  \node [below=8.6cm, font=\huge] at (atcenter){$\cong$};
  
  \draw [->] (lfunc)                     -- node [right,midway] {Some text 1} (multratcircle);
  \draw [->] (varieties)                 -- node [above,midway,sloped] {Some text 2} (multratcircle);
  \draw [->] (varieties)                 -- node [right,midway] {Some text 3} (multratbox);
  \draw [->] (multratcircle)             -- node [above,midway] {Some text 4} (multratbox);
  \draw [right hook->] (multratcircle)   -- node [right,midway] {Some text 5} (multccircle);
  \draw [right hook->] (multratbox)      -- node [right,midway] {Some text 6} (multcbox);
  \draw [->,bend right=10] (multcbox)    edge node [above,midway] {Some text 7} (multccircle);
  \draw [->,bend right=10] (multccircle) edge node [below,midway] {Some text 8} (multcbox);

  
\end{tikzpicture}
}



\emph{Rewrite the entire summary AND prelude again after finishing more of the article. Add a comment on Satake parameters for any rational multiplicative function. OR should we just remove the summary - maybe the abstract is enough? We could also add a commutative diagram involving $K_0$, L-functions, Rational multiplicative functions and Multiplicative functions in the first column, and in the second column Satake parameters, Tannakian symbols, and (not connected to Tannakian symbols) Multiplicative functions again, so that the bottom arrow is Bell derivative. Suggestion: Move the norm fixed points to a separate paper.}

Multiplicative functions are ubiquitous in modern number theory. Examples include elementary functions like the Euler totient function, the Liouville function and the M\"obius function, but also a vast class of functions constructed by reading off Dirichlet series coefficients (sometimes called Fourier coefficients) of zeta functions or L-functions with Euler products. We undertake a detailed study of the set of all (complex-valued) multiplicative functions, focussing on algebraic structures naturally occurring on this set, coming from operations such as Dirichlet convolution, natural (pointwise) product, precomposition with a power function, and various operations inspired by algebraic geometry and the theory of L-functions. 
%We have two main results, and two main applications. 

The first main result is a series of theorems equipping the set of multiplicative functions with two distinct but closely related commutative ring structures, each of which is further refined into two distinct but closely related Adams algebra structures. The term ``Adams algebra" is defined in section ??, and is essentially the same thing as Grothendieck's notion of a special lambda-ring. We define an important class of (binary and unary) operations on multiplicative functions called \emph{prime-agnostic operations}, and prove that the four lambda-ring structures taken together neatly encapsulate all prime-agnostic operations on multiplicative functions that we have been able to find in the number theory literature. The four main binary operations in question are Dirichlet convolution, unitary convolution, pointwise product, and a new operation called the tensor product. The unary operations include (but are not limited to) precomposition with a power function, the higher norm operators of Redmond and Sivaramakrishnan, as well as operations derived from base change on schemes over finite fields, and \ldots explain also for example the relation to Rankin-Selberg convolution of automorphic L-functions.

Among the four lambda-ring structures just mentioned, one is more significant than the others, and this structure is referred to as the \emph{standard} lambda-ring structure. Arithmetic geometry is to a large extent concerned with certain Tannakian categories, whose objects give rise to L-functions. Examples include motives, Galois representations, and automorphic representations. The assignment of an L-function to such an object factors through the Grothendieck ring of the Tannakian category in question. Also, to any L-function (with an Euler product) there is a naturally associated multiplicative function. Combining these observations, we get functions from various Grothendieck rings to the set of multiplicative functions, and these are all expected to be lambda-ring homomorphisms with respect to the standard lambda-ring structure (but not with respect to the other three). 

Our second main result is the construction of a new transform from the set of multiplicative functions with rational Bell series to a set of \emph{Tannakian symbols} (a new construct which by imprecise analogy relates to the theory of lambda-rings in the same way as the concept of a matrix relates to the theory of associative algebras). The point of this transform is that it allows for explicit and efficient computations in the lambda-rings just described.

As our first application, we show that our methods yield easy proofs of all known classical identities between multiplicative functions, the only exception we are aware of being the strange identities between divisor functions traditionally derived from finite-dimensionality of spaces of modular forms. %We think of this application as the beginning of a systematic theory of identities between multiplicative functions.

A second application is a series of explicit structure theorems for some of the smallest sub-lambda-rings of the standard lambda-ring structures. These sub-lambda-rings contain the classical multiplicative functions typically encountered in textbooks on elementary number theory, such as the Liouville function, the divisor functions, the Euler totient functions and the M\"obius function. The structure theorems make use of a functorial correspondence between commutative monoids and certain lambda-rings; as an example we mention that [insert example here using the additive monoid of natural numbers]. We think of these theorems as baby cases of a very general approach to structure theory for Grothendieck rings of Tannakian categories.

\emph{We could add here further comments on how Tannakian symbols provide structure and cognitive easing in the general theory of multiplicative functions, but maybe we have said enough already.}

Some of our motivational examples come from algebraic geometry and the theory of L-functions, but we want to emphasize that the main theorems and their proofs, as well as the main applications, are all of an elementary nature in the sense that they rely on nothing more than the definition of a lambda-ring, the definition of a functor, and basic facts from linear algebra and the theory of formal power series.

\subsection{Prelude}

The Laplace transform takes a problem from the world of differential equations and transforms it, almost as by magic, into an the world of algebraic equations, where the problem can be solved and the solution then translated back into the original world. The usefulness of this transform is entirely due to the fact that operations in the first world (such as convolution or differentiation) are transformed into operations in the second world which are familiar and easy to express with explicit formulas.

In a similar spirit, we develop in this paper a machine which takes an identity in the world of multiplicative functions and transforms it into an equivalent identity between \emph{Tannakian symbols}. The latter kind of identity is in almost every case far easier to prove, and the conclusion that the identity is true can be transferred back into the world of multiplicative functions.

We shall explain all this in detail, but in order to convey the flavour of our techniques, we begin by presenting three concrete examples. 

\begin{example}
The Euler totient function is defined by
$$  \varphi(n) = \textrm{the number of integers $x$ such that $1 \leq x \leq n$ and $gcd(x, n) = 1$}  $$

From this explicit definition, we can easily compute values of $\varphi(n)$ for small $n$, and we get the following table:

\vspace{6pt}
\begin{tabular}{  | c || c | c | c | c | c | c | c | c | c | c | c | c | c | c |  }
  \hline			
  $n$ & 1 & 2 & 3 & 4 & 5 & 6 & 7 & 8 & 9 & 10 & 11 & 12 & \ldots & 20  \\
  \hline
  $\varphi(n) $ & 1 & 1 & 2 & 2 & 4 & 2 & 6 & 4 & 6 & 4 & 10 & 4 & \ldots & 8  \\
  \hline  
\end{tabular}
\vspace{6pt}

Are there any patterns here? Well, there may be many, but one particular pattern can be found by choosing a number $n$ (let us choose \underline{the number 6}), and looking at all the function values of \emph{divisors} of 6. The divisors are 1, 2, 3 and 6, and the corresponding function values (circled below) are 1, 1, 2, and 2. Now take the sum of the function values. We get the number 6!

\vspace{6pt}
\begin{tabular}{  | c || c | c | c | c | c | c | c | c | c | c | c | c | c | c |  }
  \hline			
  $n$ & \bf{1} & \bf{2} & \bf{3} & 4 & 5 & \bf{6} & 7 & 8 & 9 & 10 & 11 & 12 & \ldots & 20  \\
  \hline
  $\varphi(n) $ & \raisebox{.5pt}{\textcircled{\raisebox{-.9pt} {1}}} & \raisebox{.5pt}{\textcircled{\raisebox{-.9pt} {1}}} & \raisebox{.5pt}{\textcircled{\raisebox{-.9pt} {2}}} & 2 & 4 & \raisebox{.5pt}{\textcircled{\raisebox{-.9pt} {2}}} & 6 & 4 & 6 & 4 & 10 & 4 & \ldots & 8  \\
  \hline  
\end{tabular}
\vspace{6pt}

Doing the same again, but with \underline{the number 20}, we get the divisors 1, 2, 4, 5, 10 and 20, and the corresponding function values are 1, 1, 2, 4, 4 and 8. 

\vspace{6pt}
\begin{tabular}{  | c || c | c | c | c | c | c | c | c | c | c | c | c | c | c |  }
  \hline			
  $n$ & \bf{1} & \bf{2} & 3 & \bf{4} & \bf{5} & 6 & 7 & 8 & 9 & \bf{10} & 11 & 12 & \ldots & \bf{20}  \\
  \hline
  $\varphi(n) $ & \raisebox{.5pt}{\textcircled{\raisebox{-.9pt} {1}}} & \raisebox{.5pt}{\textcircled{\raisebox{-.9pt} {1}}} & 2 &  \raisebox{.5pt}{\textcircled{\raisebox{-.9pt} {2}}} &  \raisebox{.5pt}{\textcircled{\raisebox{-.9pt} {4}}} & 2 & 6 & 4 & 6 &  \raisebox{.5pt}{\textcircled{\raisebox{-.9pt} {4}}} & 10 & 4 & \ldots &  \raisebox{.5pt}{\textcircled{\raisebox{-.9pt} {8}}}  \\
  \hline  
\end{tabular}
\vspace{6pt}

Their sum is... tadaa... 20!

We are led to believe that there is a general law here, which says that the sum of the values of the Euler function over the divisors of a given integer is precisely that given integer. This statement can be rewritten as an identity:
\begin{equation} \label{introexample1}
\sum_{d \vert n} \varphi(d) = n  
\end{equation}
There are several ways of proving this identity, but we want to give a proof that uses the new machinery introduced in this paper. We first note that $\varphi$ is a multiplicative function whose Tannakian symbol is $\frac{ \{ p  \} }{ \{  1 \}  }$, and that the identity function is also multiplicative, with Tannakian symbol $\frac{ \{ p  \} }{ \varnothing  }$. The left hand side is the M{\"o}bius transform of $\varphi$, and taking the M{\"o}bius transform of a multiplicative function corresponds to adding the Tannakian symbol $\frac{ \{ 1  \} }{ \varnothing } $. The identity (\ref{introexample1}) is therefore equivalent to the statement 
\begin{equation}
\frac{ \{ p  \} }{ \{  1 \}  } \oplus \frac{ \{ 1  \} }{ \varnothing  } = \frac{ \{ p  \} }{ \varnothing  }
\end{equation}
which is obviously true by the general rules for adding two Tannakian symbols. Hence the original identity is proved.
\end{example}


\begin{example}
Let us look at a more complicated example, this time involving an identity which was proved by ?? in ??. The $\tau$ function counts the number of positive divisors of a positive integer $n$. The table of values begins like this:

\vspace{6pt}
\begin{tabular}{  | c || c | c | c | c | c | c | c | c | c | c | c | c | c | c | c | c |  }
  \hline			
  $n$ & 1 & 2 & 3 & 4 & 5 & 6 & 7 & 8 & 9 & 10 & 11 & 12 & .. & 16 & .. & 25  \\
  \hline
  $\tau(n) $ & 1 & 2 & 2 & 3 & 2 & 4 & 2 & 4 & 3 & 4 & 2 & 6 & .. & 5 & .. & 3  \\
  \hline  
\end{tabular}
\vspace{6pt}

Now take any integer - for example \underline{the number 4}. Square the number - in our case we get 16. Now let's play a game with the divisors of 16 (which are 1, 2, 4, 8 and 16). Consider the alternating sum
$$    \tau(1) \cdot \tau(16) -  \tau(2) \cdot \tau(8) + \tau(4) \cdot \tau(4) - \tau(8) \cdot \tau(2) + \tau(16) \cdot \tau(1)  $$
If we plug in the values of the $\tau$ function here, we see that the sum is equal to $\tau(4)$.

Let's try again. We pick \underline{the number 5}, and compute
$$    \tau(1) \cdot \tau(25) -  \tau(5) \cdot \tau(5) + \tau(25) \cdot \tau(1) $$
This sum is equal to \ldots drum roll \ldots $\tau(5)$!
The pattern we see here can be generalized to any $n$, providing we take care in the placement of plus and minus signs in the sum. If we denote by $\Omega(n)$ the number of prime factors of $n$ counted with multiplicity (so that $\Omega(4) = \Omega(6) = 2$ and $\Omega(8) = 3$), then the general identity is
\begin{equation} \label{introexample2}
\sum_{d \vert n^2} (-1)^{\Omega(d)} \tau(d) \tau(\frac{n^2}{d}) = \tau(n)  
\end{equation}
Let us prove it. The $\tau$ function is multiplicative, and its Tannakian symbol is $\frac{ \{ 1 ,1 \} }{ \varnothing }$. The left hand side is what Redmond and Sivaramakrishnan \cite{} call the \emph{norm operator} applied to $\tau$, and in our new language, this is an example of an \emph{Adams operation}, for which we have a simple formula (in the world of Tannakian symbols). This particular Adams operation acts by squaring all elements of the symbol, so the left hand side of the identity is 
$$\frac{ \{ 1^2, 1^2  \} }{ \varnothing  }$$
while the right hand side is 
$$ \frac{ \{ 1, 1  \} }{ \varnothing  } $$
Now the identity follows immediately from the elementary fact that $1 \cdot 1 = 1$. 
\end{example}

\begin{example}
Finally, let us illustrate the power of our new machinery by an example of a much more general nature. We give ourselves the problem of explicitly classifying \emph{all} multiplicative functions $f$ that satisfies the identity 
\begin{equation} \label{introexample3}
\sum_{d \vert n^2} (-1)^{\Omega(d)} f(d) f (\frac{n^2}{d}) = f(n)  
\end{equation}
for all $n$. One example is the $\tau$ function discussed in the previous example. Can we do this? Yes, we can! (Let's make multiplicative functions great again!) We divide the analysis into three sections, solving the problem first in the class of \emph{all} multiplicative functions, then again the the smaller class of \emph{rational} multiplicative functions, and finally in the even smaller class of \emph{Riemann type} multiplicative functions. We rely on terminology which is only introduced later in the paper, including the \emph{Bell derivative} of a multiplicative function (Definition ??) and the Tannakian symbol of a rational multiplicative function at a prime number. 


\begin{enumerate}
\item First let's work within the entire set $Mult(\C)$ of all multiplicative functions. The identity \ref{introexample3} can be rephrased as 
$$ \adam{2}(f) = f  $$
where $\adam{2}$ is the second Adams operation in the standard lambda-ring structure on the set $Mult(\C)$. Inside this set, we are looking for the eigenspace of $\adam{2}$ which corresponds to the eigenvalue $1$. It is a general principle that a multiplicative function $f$ can be constructed by freely assigning function values at prime power arguments. The eigenvalue equation we are trying to solve imposes some restrictions on this process, which can be given a clear meaning if we focus on constructing the \emph{Bell derivative} $f'$ of $f$. In fact, the eigenvalue equation means precisely that we can freely assign values to $f'(p^e)$ for all primes $p$ and all exponents $e$ which are \underline{odd}. In other words, if we let $U$ be the set of odd positive integers and $\bbP$ be the set of prime numbers, then the process of assigning function values to the derivative gives an isomorphism between the sought-for eigenspace and the set $\C^{\bbP \times U}$ of all functions from the Cartesian product $\bbP \times U$ to the complex numbers. Any specified choice of values for $f'(p^e)$, for all pairs $(p, e)$  with $e$ odd, gives a unique multiplicative function $f$ whose values can be explicitly computed, and any such $f$ satisfies the original identity. Conversely, any $f$ satisfying the identity arises in this way. In this analysis, we have implicitly used the fact that any multiplicative function has a unique Bell derivative and a unique Bell antiderivative (and these are also multiplicative functions). 
 
This solves the problem of classifying the multiplicative functions satisfying the identity (\ref{introexample3}). The mysterious appearance of the set $U$ is immediate from the ``compression'' interpretation of the Adams operation given in Section ??.


\item Suppose we are only interested in multiplicative functions which are \emph{rational} in the sense of Definition ??. Inside this smaller set, we can describe the eigenspace in a different but equally explicit way, by using Tannakian symbols instead of the derivative. In fact, rather than ``eigenspace'' we should now use the word ``eigenmodule'', because after imposing the rationality condition, the set of functions we seek to classify is no longer a complex vector space, but rather a free $\Z$-module (with a natural grading), and to determine the functions belonging to this module, we note that at each prime, the function has a Tannakian symbol satisfying the equation 
$$   \text{Ad}^2(\frac{A}{B}) = \frac{A}{B}   $$
where $A$ and $B$ are finite multisets and $\text{Ad}^2$ is the Adams operation defined in Definition ??. Let $L$ be the set of Tannakian symbols satisfying this equation. A choice of rational multiplicative function in our eigenmodule amounts to a choice of one element of $L$ for each prime number - in other words the set of solutions is in natural bijection with the set $L^{\bbP}$ of functions from $\bbP$ to $L$. 

But what is the structure of $L$? In order to answer this question, we need a few pieces of auxiliary definitions. We define a \emph{root cycle of length $k$} to be any set of distinct non-zero complex numbers with the property the set can be ordered into a finite sequence $\alpha_1, \alpha_2, \ldots, \alpha_k$, such that $\alpha_i^2 = \alpha_{i+1}$ for $i = 1, 2, \ldots, k-1$ and $\alpha_k^2 = \alpha_1$. For example, there is only one root cycle of length $1$ (given by $\alpha_1 = 1$, and one root cycle of length 2, given by $\alpha_1 = \omega$, $\alpha_2 = \omega^2$, where $\omega$ is a primitive third root of unity. In general, every number appearing in a root cycle of length $k$ must be a (not necessarily primitive) $(2^k-1)$'th root of unity.

 Let $L_k$ be the free $\Z$-module on the set of root cycles of length $k$, and let 
$$  L = L_1 \oplus L_2 \oplus L_3 \ldots   $$
be a graded module with elements of $L_k$ placed in degree $k$. The ranks of $L_k$ (i.e. the number of root cycles of length $k$) for small values of $k$ are found in the table below:

\vspace{6pt}
\begin{tabular}{  | c || c | c | c | c | c | c | c | c | c | c | c | c |   }
  \hline			
  $k$ & 1 & 2 & 3 & 4 & 5 & 6 & 7 & 8 & 9 & 10 & 11 & 12   \\
  \hline
  $rank(L_k) $ & 1 & 1 & 2 & 3 & 6 & 9 & 18 & 30 & 56 & 99 & 186 & 335    \\
  \hline  
\end{tabular}
\vspace{6pt}

This sequence of ranks (at least up to $k=12$) happens to correspond precisely to the ranks of the degree $k$ part of a free Lie algebra on a set of generators $x_1, x_2, x_3, x_4, \ldots$ (where the generator $x_i$ is placed in degree $i$ for each positive integer $i$). This is a mystery! Is there a canonical (or natural) Lie algebra structure on our module $L$??? We do not know. If such a Lie algebra structure exists, it would transfer to a natural Lie algebra structure on a rather large class of elementary multiplicative functions. For reference, we mention that the sequence of ranks of such a free Lie algebra is labelled A059966 in OEIS (and is different from A107847, which is also related to roots of unity and agrees with the first sequence up to $k=11$). 

In any case, we have defined the module $L$, and we know that any element $v \in L$ determines a Tannakian symbol $T_v$ in the obvious way - whenever a root of unity $\alpha$ appears in a root cycle which has coefficient $m$ in the element $v$, the number $\alpha$ is included in the Tannakian symbol $T_v$ with multiplicity $m$ (i.e. $m$ copies upstairs if $m$ is positive, and $-m$ copies downstairs if $m$ is negative). Together with the explicit link between Tannakian symbols and rational multiplicative functions, this gives an explicit recipe for constructing every possible rational multiplicative function satisfying the identity (\ref{introexample3}), and as we have already mentioned, these functions are therefore in bijection with the explicit set $L^{\mathbb{P}}$.

\item Finally, if we care only for solutions in multiplicative functions of \emph{Riemann type} (meaning that its Dirichlet series can be expressed in terms of the Riemann zeta function as in Definition ??) then the eigenmodule is even smaller, as the uniformity condition implied by being of Riemann type means that the choice we made in the previous paragraph must be the same for each prime $p$. In other words, the set of solutions is in canonical bijection with the graded module $L$ described above. Of the elementary functions listed in the table of Appendix ??, we see, just by inspection of the Tannakian symbol, that the the following functions are solutions to our original puzzle:
\begin{itemize}
\item The zero function $\zero$, defined by $f(1) = 1$ and $f(n) = 0$ for all $n \geq 2$. 
%\item The constant function $\one$ defined by $f(n) = 1$ for all $n$.
\item The function $\tau_k$ for $k$ a positive integer, which by definition means the function corresponding to the Tannakian symbol $\frac{ \{1, 1, \ldots, 1 \} }{ \varnothing  }$ (with $k$ elements upstairs). For $k=1$, this is the constant function $\one$ defined by $f(n) = 1$. For $k = 2$ we get the classical $\tau$ function (counting the number of positive divisors of $n$), and for larger $k$ we get the function which counts the number of ordered factorizations of $n$ into $k$ factors.
\item The functions $\tau_{-k}$ (for $k$ a positive integer). This is the inverse (with respect to Dirichlet convolution) of $\tau_k$, and as a special case (for $k=1$), we get the The M{\"o}bius $\mu$ function.
\item The characteristic function of cubes, the characteristic function of $7$'th powers, etc (any characteristic function of $m$'th powers where $m$ is of the form $2^k-1$. 
\end{itemize}

The first three sets of functions, which together form the so-called $\tau$ family, can be viewed as the canonical copy of $\mathbb{Z}$ inside $Mult(\C)$, when the latter is equipped with the commutative ring structure defined by the operations $\oplus$ and $\otimes$ of Section 3.??. The first operation here is Dirichlet convolution and the second is an extension (in a precise sense) of the tensor product operation on automorphic representations or motives.

%Among the functions which are \emph{not} solutions to our eigenvector problem, we find for example the Euler $\varphi$ function, the divisor functions $\sigma_k$ for $k \geq 1$, etc.

%\emph{Add the remaining classical functions and emphasize again that we determine if the function solves the eigenvalue problem or not simply by inspecting the Tannakian symbol (no computation involved!)}

\end{enumerate}


\end{example}



\subsection{Acknowledgements}

This article is an extension of a high-school project carried out by the first and third author, and all of us (two students and one teacher) would like to thank Fagerlia Upper Secondary School in {\AA}lesund, Norway, and in particular Ivar Karsten Lerstad and Yngve Omen{\aa}s, for encouragement and good working conditions. We are also indebted to many students, and specifically want to thank Fride Nordstrand Nilsen and Charlotte Lund-Hanssen for their keen interest in zeta functions, Pauline Schulze for making mathematics popular in {\AA}lesund and for inspirational writing about a lemma of Mochizuki, and finally Magnus and Olav Hellebust Haaland for their Sage implementation of the Berlekamp-Massey algorithm, which we have used in many of our computations. 

The second author still feels indebted to Lars Svensson, whose legendary undergraduate lectures in Stockholm gave him the impression that to include the ZFC axioms, modular forms, topos theory and Clifford algebras in a first-year undergraduate calculus course is a perfectly natural thing to do. Without you Lars as a role model, I would probably never have assigned a project on lambda-rings and multiplicative functions to two of my ambitious high-school students.

When it comes to mathematical references, we would like to acknowledge the book of McCarthy \cite{} the Online Encyclopedia of Integer Sequences, and the excellent (but unpublished) survey paper of Mathar \cite{}. These three sources provided us with almost all the examples upon which our results are based.

As emphasized by McCarthy, the history of multiplicative functions is ripe with discovery and rediscovery, and it is possible that some specific cases of the results presented here are already in the literature. The existence of our two commutative ring structures on $Mult(\mathbb{C})$ is almost certainly known, but we have not been able to find a reference. We believe that the existence of natural lambda-ring structures on multiplicative functions is not mentioned anywhere in the literature, but Peter Arndt pointed out to us that Jesse Elliott gave a talk in 2013 \cite{} in which the existence of a single such lambda-ring structure was mentioned. However, Elliott does not seem to relate this lambda-ring structure to previously studied operations from number theory or algebraic geometry.

Say something about Unge Forskere and the original report here. Mention also CICM paper perhaps, and current work in progress by the authors, including automated conjecture-making. 

Say something here about lambda-rings and rational Witt vectors. Give reference to Borger, Grinberg, Yau, and perhaps Scholze-Kucharczyk. Thanks to Marcus Zibrowius for exquisite asparagus and for helping us with formulas for symmetric powers.


\subsection{Notation and terminology}

Throughout the paper, we write $\mathbb{N}$ for the set of \emph{strictly positive} integers. The symbols $\mathbb{Z}$, $\mathbb{Q}$, $\mathbb{R}$, and $\mathbb{C}$ all carry their usual meaning. We use $\mathbb{P}$ to denote the set of prime numbers, and we introduce the notation
$$ \mathbb{PP} = \{ 2, 3, 4, 5, 7, 8, 9, 11, 13, 16, \ldots  \}  $$
for the set of all prime powers, i.e. integers that can be written on the form $p^e$, with $p \in \mathbb{P}$ and $e \in \mathbb{N}$.

When $X$ and $Y$ are sets, we write $Y^X$ for the set of all functions from $X$ to $Y$. If $Y$ in this situation carries some algebraic structure, the set $Y^X$ inherits the same kind of algebraic structure by pointwise operations, and we shall refer to this as the \defhl{pointwise} algebraic structure on $Y^X$.

We write $gcd(m, n)$ or sometimes simply $(m, n)$, for the greatest common divisor of two integers $m$ and $n$. We write $lcm(m, n)$ or simply $[m ,n]$ for their least (strictly positive) common multiple.

If $X$ is a finite set, we write $\# X$ for the number of elements in the set $X$. 


With a few exceptions, we shall systematically employ notation according to the list below. Only the letter $h$ serves double functions - this should not create any confusion.
\todo{Review this list of notation after finishing article.}
\begin{itemize}
\item $f$, $g$, $h$: Unspecified multiplicative functions.
\item Various Greek letters: Specific multiplicative functions.
\item $m$, $n$: Natural numbers, either given as arguments to a multiplicative function or used in some other context.
\item $k$ (and occasionally also $h$): Integer parameter indexing a family of multiplicative functions or a family of lambda-ring operations (Example: The divisor functions $\sigma_k$).
\item $r$ will be a real parameter, and $z$ will be a complex parameter.
\item $p$ will always denote a prime, and $e$ will be an exponent. A prime power will be denoted by $p^e$ or occasionally by the letter $q$.
\item $X$ and $Y$ will be sets, sometimes perhaps equipped with some algebraic structure. (\emph{Are these letters actually used??}
\item Notation for monoids? $M$ and $N$?
\item $A$, $B$, $C$ and $D$ will be multisets.
\item $\alpha_i$ and $\beta_j$ will be elements (usually complex numbers) of some multiset. Usually $i$ will go from 1 to $m$ and $j$ will go from $1$ to $n$.
\item Explain convention that elements from $A$ belongs to the denominator etc???
\item Coefficients of a power series $c_1$, $c_2$, etc, or do we also need $a_i$ and $b_i$? Or some other notation, like capital letters???
\item $t$ will be the formal variable in a formal power series, and $s$ will be the formal variable in a formal Dirichlet series.
\item $R$ will be a ring. Sometimes we shall use $Z$ and $Q$, explain why.
\item Given $R$ (ring or semiring), we write $R_{add}$ and $R_{mult}$ for the underlying monoids of $R$. I think this is best, then we can add a condition as superscript, like in $\mathbb{R}_{add}^{\geq 0}$.
\item If $f$ is a multiplicative function, its Bell derivative will be denoted by $f'$ or sometimes by $\BD f$. 
\item Symmetric operations will be denoted by three- or four-letter abbreviations, such as $Ext$, $Sym$, etc (review this decision and add newcommands!!)
\item $T$ and $S$ will be Tannakian symbols.
\item Surely there are letters we have forgotten. 
\item $a$, $b$, $c$ will be elements of the multisets appearing in Tannakian symbols.
\end{itemize}



Explain the notation and terminology related to "box" and "circle"?

Finally, we mention that in addition to Appendix ?? on Adams operations, we have also included appendices ??, ?? and ?? which together give an overview of the plethora of functions, operations, classes of functions and classes of operations. For a serious reader it might be helpful to keep a print copy of all these appendices at hand while reading the remainder of the article.

\emph{Fix conventions for font styles of functors, operations on power series, operations on lambda-rings, operations of multiplicative functions (maybe not) and the multiplicative functions themselves.}

A few of the operations on multiplicative functions which we found in the literature did not have any names in the sources we consulted, and in these cases we have introduced names which seem reasonable. Same is true for operations, and this concerns the Schemmel transform, the LCM convolution, the GCD transform, the $k-$twisted convolution and ???

The most challenging aspect of choosing notation is that there are so many Adams operations around. Explain double usage??? Print appendix!


We now collect the definitions we use from abstract algebra:

Define relevant categories?

Add here brief intro to monoids/commutative monoids. Rings/commutative rings. Follow CICM paper perhaps. Or refer to other sources and skip this stuff here.

Define binomial coefficient, define binomial ring?

Explain what it means for a map to commute with an operation.


\subsection{Prerequisites}

Some of our constructions are inspired by notions from algebraic geometry (such as schemes, varieties, base change and Weil restriction) and notions from representation theory and analytic number theory (such as automorphic L-functions, Tannakian categories, and the canonical lambda-ring structure on the Grothendieck ring of any Tannakian category). From a logical point of view however, the definitions and results of the present paper do not rely on any material from these fields of study beyond a superficial familiarity with the words ``category'' and ``functor''.

We hope that the structure of the presentation makes all the results accessible to anyone with some basic knowledge of elementary number theory and abstract algebra, while at the same time making clear to experts (by way of examples) what the connections are to algebraic geometry and to the theory of motivic and automorphic L-functions. 

In order to explain the connections to algebraic geometry without relying on the full machinery of commutative algebra and locally ringed spaces, we have presented a small didactical experiment in Section ??, where we introduce the notion of a \emph{micro-scheme}. A micro-scheme is any machine that takes a finite ring as input and gives a finite set as output. These machines may serve as a high-school teacher's substitute for schemes in the sense of Grothendieck.

\subsection{The scope of this article}

Goals: (1) Language and (2) Application to algebraic structure on multiplicative functions.

Leaving out: (1) Details on many other applications (see future publications), (2) Rings which are not subrings of $\C$, and (3) Metric structures.

Explain/justify why we work with complex-valued functions? Domain? Field? Alg closed? Q-algebra? Give ref to general discussion in endnotes. Mention somewhere p-adic modular forms/p-adic L-functions, and also dual numbers and Tao's ideas, as possible applications of more general target rings.

Mention that we introduce many topics which we intend to treat in more detail in future work. In particular, connections to the ideas of Tao on derived multiplicative functions.

Mention also the Endnotes, where we sum up what remains to be done.


